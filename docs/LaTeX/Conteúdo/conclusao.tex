\chapter{Conclusão}
\paragraph{}

% Em suma, este trabalho representou uma valiosa contribuição para aprofundar nosso conhecimento sobre a área de migração de dados e exploração de bases de dados NoSQL, oferecendo uma visão prática das estratégias e desafios envolvidos. Para trabalhos futuros, sugerimos a otimização contínua das consultas, a expansão das funcionalidades do sistema, como, por exemplo, a inclusão mais do que uma unidade hospitalar.

% Foi possível observar que a flexibilidade e escalabilidade dos sistemas NoSQL proporcionaram uma gestão mais eficiente dos dados, especialmente em termos de consultas e operações complexas. A utilização da biblioteca APOC em Neo4j e a definição de \textit{constraints} foram essenciais para manter a integridade dos dados durante a migração. Além disso, as consultas implementadas demonstraram capacidades operacionais superiores em comparação com o sistema relacional original, facilitando a obtenção de informações detalhadas.

% Para trabalhos futuros, sugerimos a otimização contínua das consultas para melhorar ainda mais o desempenho do sistema. Além disso, a expansão das funcionalidades do sistema pode abranger a integração de novos tipos de dados, a implementação de novos \textit{triggers}, e a melhoria das interfaces de consulta. Adicionalmente, a aplicação dos conhecimentos adquiridos em outros domínios de dados pode oferecer \textit{insights} valiosos para a construção de sistemas similares. Esta abordagem poderá contribuir para o desenvolvimento de soluções mais robustas e adaptáveis às necessidades específicas de diferentes indústrias.

O trabalho realizado sobre a migração e exploração de bases de dados NoSQL representou um avanço significativo no conhecimento e aplicação dessas tecnologias em ambientes hospitalares. A migração de um sistema baseado em Oracle SQL para MongoDB e Neo4j não só demonstrou a viabilidade e os benefícios dos sistemas NoSQL, como também ofereceu \textit{insights} práticos sobre as estratégias e desafios enfrentados.

Observou-se que a flexibilidade e escalabilidade dos sistemas NoSQL proporcionaram uma gestão de dados mais eficiente, particularmente em consultas e operações complexas. Em relação ao MongoDB a utilização da ferramenta Atlas foi essencial para a criação de \textit{triggers} essenciais ao nosso modelo de dados. Quanto a Neo4j a utilização da biblioteca APOC e a definição de \textit{constraints} foram cruciais para manter a integridade dos dados durante a migração. Além disso, as consultas implementadas exibiram capacidades operacionais superiores em comparação com o sistema relacional original, facilitando a obtenção de informações detalhadas.

Para futuros trabalhos, recomenda-se a otimização contínua das consultas para aprimorar ainda mais o desempenho do sistema. A expansão das funcionalidades do sistema pode incluir a integração de novos tipos de dados, a implementação de novos \textit{triggers}, e a melhoria das interfaces de consulta. 

Esta abordagem não apenas reforça a versatilidade e a capacidade dos sistemas NoSQL, mas também destaca a importância de um contínuo processo de inovação e adaptação às demandas crescentes e complexas dos dados modernos, particularmente em contextos críticos como o hospitalar.