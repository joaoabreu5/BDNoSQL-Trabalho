\chapter{Introdução}
\paragraph{}
No âmbito da unidade curricular de Bases de Dados NoSQL inserida no Mestrado em Engenharia Informática foi desenvolvido um projeto prático com o objetivo de explorar e aplicar diferentes paradigmas de bases de dados. Este projeto tem como foco a migração de dados de um sistema de gestão hospitalar baseado em \textbf{SQL} (Oracle SQL) para dois sistemas não relacionais distintos: \textbf{MongoDB}, uma base de dados orientada a documentos, e \textbf{Neo4j}, uma base de dados orientada a grafos.

A importância desta migração reside na necessidade de adaptar os sistemas de base de dados às crescentes demandas e complexidades dos dados modernos. Enquanto as bases de dados relacionais tradicionais oferecem uma estrutura rígida e tabular, as bases de dados NoSQL proporcionam flexibilidade e escalabilidade para lidar com volumes massivos de dados, estruturas variadas e relações complexas.

Ao longo deste trabalho, foram definidos processos para a migração eficaz dos dados, a implementação de consultas para demonstrar as capacidades operacionais dos sistemas implementados e uma análise crítica comparando as funcionalidades dos sistemas não relacionais com o sistema relacional original.

Este relatório técnico apresenta de forma detalhada o trabalho desenvolvido, desde a análise inicial até à implementação final dos sistemas de base de dados não relacionais, evidenciando as estratégias adotadas, os desafios enfrentados e os resultados obtidos.