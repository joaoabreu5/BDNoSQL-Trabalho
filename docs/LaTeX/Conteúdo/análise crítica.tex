\chapter{Análise Crítica}
\paragraph{}

% Ao longo do projeto, desenvolvemos a migração de dados de um sistema de gestão hospitalar baseado em \textbf{Oracle SQL} para sistemas NoSQL, especificamente \textbf{MongoDB} e \textbf{Neo4j}. Esta migração foi crucial para atender às crescentes demandas e complexidades dos dados modernos, proporcionando maior flexibilidade e escalabilidade.

%Ao nível da exploração, esta foi levada ao detalhe. Ao longo de todo o projeto, para cada uma das bases de dados — relacional, orientada a documentos ou orientada a grafos — desenvolvemos todas as queries, funções e outras operações de forma a serem utilizadas e aplicadas no contexto real de um centros hospitalares.

%Esforçámo-nos por garantir que estas ferramentas pudessem ser usadas eficazmente tanto pelos pacientes, através da utilização de perfis personalizados, como pelo staff hospitalar para a gestão de informações e operações diárias. Adicionalmente, desenvolvemos funcionalidades específicas para a gestão de episódios médicos, tanto os que estão a decorrer como os já passados, assegurando uma administração eficiente e integrada dos dados hospitalares.

%Este nível de detalhe na exploração permitiu-nos criar um sistema robusto e versátil, capaz de atender às necessidades de diversos utilizadores num ambiente hospitalar real.

%Durante o desenvolvimento, enfrentámos diversos desafios, como a manutenção da integridade dos dados e a adaptação de triggers. Para resolver esses problemas, utilizámos a biblioteca APOC em Neo4j para a criação de triggers e definimos constraints para garantir a unicidade dos dados.

%- Falar da extensão exploração realizada em cada um das DB

%- Falar dos Triggers em mais detalhe

%- Falar do Counter

%- Falar das Datas?!

Ao longo deste projeto, realizámos a migração de dados de um sistema de gestão hospitalar baseado em Oracle SQL para sistemas NoSQL, nomeadamente MongoDB e Neo4j. Esta migração revelou-se essencial para responder às crescentes exigências e complexidades dos dados modernos, oferecendo maior flexibilidade e escalabilidade.

A exploração das bases de dados foi conduzida de forma meticulosa. Para cada uma das bases de dados — relacional, orientada a documentos ou orientada a grafos — desenvolvemos consultas, funções e outras operações de modo a serem aplicáveis no contexto real de um hospital. Em MongoDB, implementámos várias funções e expressões para obter informações detalhadas e gerais. Em Neo4j, as consultas foram otimizadas para obter dados complexos de forma eficiente.

Procurámos garantir que estas ferramentas pudessem ser usadas de forma eficaz tanto pelos pacientes, através de perfis personalizados, como pelo pessoal hospitalar para a gestão de informações e operações diárias. Adicionalmente, desenvolvemos funcionalidades específicas para a gestão de episódios médicos, tanto os que estão a decorrer como os já finalizados, assegurando uma administração eficiente e integrada dos dados hospitalares.

Este nível de detalhe na exploração permitiu-nos criar um sistema robusto e versátil, capaz de atender às necessidades de diversos utilizadores num ambiente hospitalar real.

Durante o desenvolvimento, enfrentámos vários desafios, como a manutenção da integridade dos dados e a adaptação de \textit{triggers}. Os \textit{triggers} desempenharam um papel crucial na manutenção da integridade dos dados e na automatização de processos. A criação de \textit{triggers} em Neo4j, com o auxílio da biblioteca APOC, e em MongoDB, recorrendo à ferramenta Atlas, foi fundamental para a geração automática de identificadores únicos e a preservação da consistência dos dados.

Introduzimos também um nó auxiliar (\textit{Counter}) e uma coleção auxiliar (\textit{counters}) cuja principal função é armazenar o número máximo dos identificadores utilizados até ao momento. Estas duas abordagens facilitaram a geração de novos identificadores únicos para futuros elementos das nossas bases de dados. A gestão de datas foi abordada com cuidado para garantir a precisão e a consistência dos registos temporais nos diferentes sistemas de bases de dados.